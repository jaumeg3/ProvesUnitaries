\documentclass[11pt]{article}
\usepackage{ragged2e}
\usepackage[utf8]{inputenc}
\usepackage[catalan]{babel}
\usepackage{graphicx}
\usepackage{array}
\usepackage{afterpage}
\usepackage{listings}
\usepackage{caption}
\graphicspath{ {Images/} }
\usepackage{chngcntr}
\usepackage{hyperref}
\usepackage{lscape}
\usepackage{enumitem}
\usepackage{color}
\definecolor{aurometalsaurus}{rgb}{0.43, 0.5, 0.5}
\graphicspath{ {Images/} }
\newcolumntype{L}{>{\arraybackslash}m{2.5cm}}

\usepackage[a4paper, total={6in, 8in}]{geometry}

\begin{document}
\begin{titlepage}

\newcommand{\HRule}{\rule{\linewidth}{0.5mm}} % Defines a new command for the horizontal lines, change thickness here

\center % Center everything on the page

%----------------------------------------------------------------------------------------
%	HEADING SECTIONS
%----------------------------------------------------------------------------------------

\textsc{\LARGE Universitat de Lleida}\\[1cm] % Name of your university/college
\includegraphics[scale=0.75]{Images/eps.png}\\[0.65cm] % Include a department/university logo - this will require the graphicx package
\textsc{\Large Grau en Enginyeria Informàtica}\\[0.3cm] % Major heading such as course name
\textsc{\large Enginyeria del Programari}\\[0.5cm] % Minor heading such as course title

%----------------------------------------------------------------------------------------
%	TITLE SECTION
%----------------------------------------------------------------------------------------

\HRule \\[0.4cm]
{\huge \bfseries Tercera pràctica d'Enginyeria del Programari}\\[0.0cm]
% Title of your document
\HRule \\[1cm]

%----------------------------------------------------------------------------------------
\begin{minipage}{0.4\textwidth}
\begin{flushleft} \large
\emph{Autors:}\\
Jaume Giralt Barbé
\end{flushleft}
\end{minipage}
~
\begin{minipage}{0.4\textwidth}
\begin{flushright} \large
\emph{Professor:} \\
Juan Manuel Gimeno Illa
% Supervisor's Name
\end{flushright}
\end{minipage}\\[4cm]

%----------------------------------------------------------------------------------------
%	DATE SECTION
%----------------------------------------------------------------------------------------
{\large \today}\\[3cm] % Date, change the \today to a set date if you want to be precise
\vfill % Fill the rest of the page with whitespace
\end{titlepage}
\newpage
\tableofcontents
%\listoftables
%\listoffigures
\clearpage
\newpage
\justify
\section{Contingut}
Per la realització d'aquesta pràctica he dividit el contingut amb dos directoris. El directori \textit{src} i el directori \textit{tests}. En el directori \textit{src} s'hi poden trobar els fitxers escrits amb llenguatge java  que simulen el funcionament d'un sistema de votació electrònica. No he realitzat la part opcional de introduir el sistema de autenticació del iris per fer una mala planificació del temps a la hora de realitzar la pràctica.\\
En el directori \textit{tests} estan implementats els jocs de tests per provar la pràctica. En aquest document intentaré explicar el que he volgut provar per cada classe i/o mètodes del sistema.\\\\
També hi ha la carpeta \textit{.git} on es poden veure els diferents \textbf{commits} usant un sistema de control de versions. També he pujat la pràctica en el meu servei de emmagatzemament de repositoris \textbf{GitHub}\footnote{Git : \url{https://github.com/jaumeg3/ProvesUnitaries}}.\\\\
Seguidament explicaré els tests per cada classe.
\begin{center}
	\begin{figure}[!ht]
		\includegraphics[scale=1]{Images/git.jpg}
	\end{figure}
\end{center}
\newpage
\section{Explicació dels tests}
	\subsection{Paquet data}
	\subsubsection{Mètodes test del paquet data}
	Els mètodes del paquet data serveixen per emmagatzemar dades només. Hi han quatre classes en aquest paquet que són \textit{Vote, Party, MailAddress i Signature}. Aquestes classes emmagatzemen un String excepte Signature que emmagatzema un array de bytes. Per tant, els tests que he fet són similars.
	\\
	\begin{lstlisting}[basicstyle=\ttfamily\scriptsize,language=Java]
public class VoteTest {
   	@Test
   	public void testToString() throws Exception {
    			Vote vote = new Vote("");
    			String expected = "Vote{option=''}";
    			assertEquals(expected,vote.toString());
   	}

   	@Test
   	public void testEquals() throws Exception {
    			Vote one = new Vote("");
    			Vote two = new Vote("");
    			assertTrue(one.equals(two));
   	}
   		
   	@Test
   	public void testHashCode() throws Exception {
    			Vote one = new Vote("");
    			Vote two = new Vote("");
    			assertTrue(one.hashCode() == two.hashCode());
   	}
}   	
	\end{lstlisting}
	Com podem veure he fet tres tests.
	\begin{itemize}
		\item \textbf{testToString:} En aquest test creo un vot nou. Posteriorment creo un String del que espero que surti quan executi el mètode toString de la classe i posteriorment comprovo amb el mètode assertEquals si els dos Strings són iguals.
		\item \textbf{testEquals:} En aquest test creo dos instàncies de vot i després utilitzo el mètode equals de la classe.
		\item \textbf{testHashCode:} En aquest test també creo dues instàncies de vot i posteriorment comprovo que el hashCode siguin idèntics.
	\end{itemize}
	En el mètode Signature, la diferència és que al ser un array de bytes he de passar el string a un array de bytes cridant la funció \textit{GetBytes()}.
\newpage
\subsection{Paquet Kiosk}
\subsubsection{Classe Activate Emission Test}
\begin{lstlisting}[basicstyle=\ttfamily\scriptsize,language=Java]
public class ActivateEmissionTest {

    @Test
    public void TestGoodActivateEmission() throws Exception {
        TrueValidationServiceMock TrueValidation = new TrueValidationServiceMock();
        VotingMachine votingMachine = new VotingMachine();
        ActivationCard activationCard = new ActivationCard("Any");

        votingMachine.setValidationService(TrueValidation);
        votingMachine.activateEmission(activationCard);
        assertTrue(votingMachine.canVote());
    }

    @Test(expected = IllegalStateException.class)
    public void TestActivateEmission2() throws Exception {
        FalseValidationServiceMock falseValidation = new FalseValidationServiceMock();
        VotingMachine votingMachine = new VotingMachine();
        ActivationCard activationCard = new ActivationCard("Any");

        votingMachine.setValidationService(falseValidation);
        votingMachine.activateEmission(activationCard);
    }

    @Test(expected = IllegalStateException.class)
    public void TestActivateEmission3() throws Exception {
        TrueValidationServiceMock TrueValidation = new TrueValidationServiceMock();
        VotingMachine votingMachine = new VotingMachine();
        ActivationCard activationCard = new ActivationCard("Any");

        votingMachine.setValidationService(TrueValidation);
        votingMachine.activateEmission(activationCard);
        votingMachine.activateEmission(activationCard);
    }
	\end{lstlisting}
	En aquest test comprovaré que el mètode Activate Emission de la classe VotingMachine funcioni correctament. Mirant la implementació del mètode vaig veure convenient fer un test on funciones correctament, on el servei de validació de la targeta em donés false i on la màquina ja estigues activada. En el primer test comprovo que tot funciona correctament. Creo un doble per a que em retorni que la targeta d'activació és vàlida i activo la targeta. Després comprovo que pugui exercir el dret a vot. En els altres test primer faig que falli la validació retornant que qualsevol targeta retorni false en el doble i per tant llenci una excepció i que la maquina de votació ja estigui activa intentant activar-la dos cops.
	\newpage
	\subsubsection{Classe Activation Card Test}
	\begin{lstlisting}[basicstyle=\ttfamily\scriptsize,language=Java]
public class ActivationCardTest {

    private ActivationCard testingCard;

    @Before
    public void setUp() throws Exception {
        this.testingCard = new ActivationCard("00000");
    }

    @Test
    public void eraseTest() throws Exception {
        System.out.print("erase Test");
        this.testingCard.erase();
        assertTrue(!this.testingCard.isActive());
    }

    @Test
    public void isActiveTest() throws Exception {
        System.out.print("isActive Test");
        assertTrue(this.testingCard.isActive());
    }

    @Test
    public void getCodeTest() throws Exception {
        System.out.print("getCode Test");
        assertEquals("00000",this.testingCard.getCode());
    }

    @Test
    public void wrongGetCodeTest() throws Exception {
        System.out.print("Wrong getCode Test");
        String wrong = "00001";
        assertEquals(1 , wrong.compareTo(this.testingCard.getCode()));
    }

    @Test
    public void wrongEqualsTest() throws Exception {
        System.out.print("Wrong Equals Test");
        assertTrue(!this.testingCard.equals(new ActivationCard("00001")));
    }

    @Test
    public void goodEqualsTest() throws Exception {
        System.out.print("Good Equals Test");
        assertTrue(this.testingCard.equals(new ActivationCard("00000")));
    }

}
	\end{lstlisting}
	En aquesta classe provaré les funcionalitats de la classe Activation Test. Comprovaré que funcionin els mètodes isActive() i erase(). Per fer-ho tinc una variable booleana en el codi que m'indica si la targeta està activa o no. Si elimino la targeta, el codi queda desactivat. Per tant, el isActive em retornarà fals i si creo la targeta i ho comprovo si està activa em retornarà true. Per fer el getCode i el equals creo una targeta i comprovo amb codis diferents i/o iguals que em doni el resultat desitjat.
\newpage
	\subsubsection{Can Vote Test}
	\begin{lstlisting}[basicstyle=\ttfamily\scriptsize,language=Java]
public class CanVoteTest {
    private VotingMachine votingMachine;
    private TrueValidationServiceMock trueValidationServiceMock;
    private ActivationCard card;

    @Before
    public void setUp() throws Exception {
        this.votingMachine = new VotingMachine();
        this.trueValidationServiceMock = new TrueValidationServiceMock();
        this.card = new ActivationCard("AnyCode");
    }

    @Test
    public void TestGoodCanVote() throws Exception {
        this.votingMachine.setValidationService(trueValidationServiceMock);
        this.votingMachine.activateEmission(this.card);
        assertTrue(this.votingMachine.canVote());
    }

    @Test
    public void TestBadCanVote() throws Exception {
        assertTrue(!this.votingMachine.canVote());
    }

}
	\end{lstlisting}
	En aquesta classe de test comprovaré el correcte funcionament del mètode canVote de la classe VotingMachine. Per comprovar el mètode només tenim la possibilitat de comprovar si la maquina està o no activada. En cas de estar activa es podrà votar i en cas que no no es podrà votar. Per comprovar he necessitat implementar un doble que em retorni sempre true al validar la targeta. En el primer test, faig un activate emission i per tant activo la maquina i en el segon intento votar sense haver fet un activate emission i per tant em retorna una excepció.
	\\\\
	\subsubsection{Send Receipt Test}
	\begin{lstlisting}[basicstyle=\ttfamily\scriptsize,language=Java]
public class SendReceiptTest {
    private VotingMachine votingMachine;
    private MailAddress mailAddress;
    private ActivationCard card;
    private TrueValidationServiceMock trueValidation;
    private MailerServiceImpl mailerService;
    private SignatureServiceImpl signatureService;


    @Before
    public void setUp() throws Exception {
        this.votingMachine = new VotingMachine();
        this.card = new ActivationCard("AnyCode");
        this.mailAddress = new MailAddress("AnyAddress");
        this.trueValidation = new TrueValidationServiceMock();
        this.mailerService = new MailerServiceImpl();
        this.signatureService = new SignatureServiceImpl();
    }

    @Test(expected = IllegalStateException.class)
    public void TestBadMachineSendReceipt() throws Exception {
        this.votingMachine.setValidationService(this.trueValidation);
        this.votingMachine.sendReceipt(this.mailAddress);
    }

    @Test(expected = IllegalStateException.class)
    public void TestBadHasVotedSendReceipt() throws Exception {
        this.votingMachine.setValidationService(this.trueValidation);
        this.votingMachine.activateEmission(this.card);
        this.votingMachine.sendReceipt(this.mailAddress);
    }

    @Test
    public void TestGoodSendReceipt() throws Exception {
        Vote vote = new Vote("AnyVote");
        this.votingMachine.setValidationService(this.trueValidation);
        this.votingMachine.setSignatureService(this.signatureService);
        this.votingMachine.setMailerService(this.mailerService);
        this.votingMachine.setVotesDB(new VotesDBImpl());
        this.votingMachine.setVotePrinter(new VotePrinterImpl());

        this.votingMachine.activateEmission(this.card);
        this.votingMachine.vote(vote);
        this.votingMachine.sendReceipt(this.mailAddress);
        assertEquals(this.mailAddress,this.mailerService.address);
        Signature expect = this.signatureService.sign(vote);
        assertEquals(expect, this.mailerService.signature);
        assertTrue(this.mailerService.status);
    }
}
	\end{lstlisting}
	En aquesta classe de test comprovaré el mètode send. Hi han tres possibles tests que seria que falles per culpa que no s'ha votat encara o que la maquina encara estigui connectada i preparada per votar i vulguis enviar el rebut del vot.L'altra opció es que tot sigui correcte. Per fer això necessitem un doble que ens retorni sempre vàlid el codi. En els dos primers test assegurem que falli i en l'ultim comprovem que els atributs es guardin en el Mail Service implementation.
	\\\\
	\subsubsection{Vote Test}
	\begin{lstlisting}[basicstyle=\ttfamily\scriptsize,language=Java]
public class VoteTest {
    private Vote vote;
    private VotingMachine votingMachine;

    @Before
    public void setUp() throws Exception {
        this.vote = new Vote("AnyParty");
        this.votingMachine = new VotingMachine();
    }

    @Test(expected = IllegalStateException.class)
    public void TestBadVote() throws Exception {
        this.votingMachine.vote(this.vote);
    }

    @Test
    public void TestGoodVote() throws Exception {
        ActivationCard card = new ActivationCard("Any");
        TrueValidationServiceMock trueValidationServiceMock = new TrueValidationServiceMock();
        VotePrinterMock votePrinter = new VotePrinterMock();
        VotesDBImpl votesDB = new VotesDBImpl();
        this.votingMachine.setVotePrinter(votePrinter);
        this.votingMachine.setVotesDB(votesDB);
        this.votingMachine.setValidationService(trueValidationServiceMock);
        this.votingMachine.activateEmission(card);
        this.votingMachine.vote(this.vote);
        List<Vote> lst = new ArrayList<>();
        lst.add(this.vote);
        assertEquals(votesDB.getVotes(),lst);
        assertEquals(votePrinter.outContentMock.toString(),vote.toString());
        assertEquals(votePrinter.errContentMock.toString(),"");
        assertTrue(votingMachine.hasVoted);
    }
}
	\end{lstlisting}
	En aquesta classe del mètode vote comprovem l'acció principal del sistema que és efectuar el vot. Només hi han dos possibles situacions, o pots votar o no. Per tant, per provar que no podem votar crearem un voting machine i intentarem votar sense haver activat la maquina. L'altra possible situació es que funciona i per tant comprovarem que s'enregistra bé el vot i que s'imprimeix correctament el vot. Després comprovem que hem votat.
	\newpage
	\subsection{Paquet services}
	\subsubsection{Mailer Service Implementation Test}
	\begin{lstlisting}[basicstyle=\ttfamily\scriptsize,language=Java]
public class MailerServiceImplTest {

    @Test
    public void TestSend() throws Exception {
        MailAddress address = new MailAddress("AnyAddress");
        Signature signature = new Signature("Any".getBytes());
        MailerServiceImpl mailerService = new MailerServiceImpl();
        mailerService.send(address,signature);
        assertEquals(address,mailerService.address);
        assertEquals(signature,mailerService.signature);
        assertTrue(mailerService.status);
    }

}
	\end{lstlisting}
	En aquest test comprovo que funcioni correctament la implementació del Mail Service. Creo els paràmetres necessaris per inicialitzar el Mail Address i després comprovo que els paràmetres que he introduït són iguals que els que s'han guardat en l'objecte Mail Address creat.
	\\\\
	\subsubsection{Signature Service Implementation Test}
	\begin{lstlisting}[basicstyle=\ttfamily\scriptsize,language=Java]
public class SignatureServiceImplTest {
    private Vote vote;
    private SignatureServiceImpl signatureService;

    @Before
    public void setUp() throws Exception {
        this.vote = new Vote("any");
        this.signatureService = new SignatureServiceImpl();
    }

    @Test
    public void TestSign() throws Exception {
        byte[] expected = this.vote.toString().getBytes();
        assertEquals(new Signature(expected), this.signatureService.sign(this.vote));
    }

}
	\end{lstlisting}
	En aquesta classe provarem que funciona correctament el signature service. En aquest test comprovarem que la signatura d'un test és faci correctament. Per això, inicialitzarem els paràmetres necessaris i crearem l'objecte. Després comprovarem que la signatura que ens retornarà l'objecte coincideixi amb l'esperada.
	\newpage
	\subsubsection{Validate Service Implementation Test}
	\begin{lstlisting}[basicstyle=\ttfamily\scriptsize,language=Java]
public class ValidationServiceImplTest {
    private ActivationCard card;
    private ValidationServiceImpl validationService;

    @Before
    public void setUp() throws Exception {
        this.card = new ActivationCard("anyCode");
        this.validationService = new ValidationServiceImpl();
    }

    @Test
    public void TestValidate() throws Exception {
        assertTrue(this.validationService.validate(this.card));
    }

    @Test
    public void TestDeactivate() throws Exception {
        this.validationService.deactivate(this.card);
        assertTrue(!this.validationService.validate(this.card));
    }

}
	\end{lstlisting}
	En aquest test comprovarem que la implementació del Validate Service ens funcioni tal com volem. Per això provarem que quan es validi una targeta el seu estatus sigui true i després provarem que si la desactivem, el seu estatus sigui desactivat.
	\\\\
	\subsubsection{Vote Printer Implementation Test}
	\begin{lstlisting}[basicstyle=\ttfamily\scriptsize,language=Java]
public class VotePrinterImplTest {
    private VotePrinterImpl votePrinter;
    private Vote vote;
    private final ByteArrayOutputStream outContent = new ByteArrayOutputStream();
    private final ByteArrayOutputStream errContent = new ByteArrayOutputStream();

    @Before
    public void setUp() throws Exception {
        this.votePrinter = new VotePrinterImpl();
        this.vote = new Vote("AnyParty");
        System.setOut(new PrintStream(outContent));
        System.setErr(new PrintStream(errContent));

    }

    @Test
    public void TestPrint() throws Exception {
        this.votePrinter.print(this.vote);
        assertEquals("AnyParty",this.outContent.toString());
        assertEquals("",this.errContent.toString());
    }

}
	\end{lstlisting}
	En aquesta classe comprovarem la funció print de la implementació del Vote Printer. Redireccionarem la sortida per defecte i la sortida d'errors i comprovarem que l'output sigui igual al string que esperàvem i també que la sortida d'errors sigui nul·la.
	\\\\
	\subsubsection{Votes DB Implementation Test}
	\begin{lstlisting}[basicstyle=\ttfamily\scriptsize,language=Java]
public class VotesDBImplTest {
    private VotesDBImpl votesDB;
    private Vote vote;
    private List<Vote> expected;

    @Before
    public void setUp() throws Exception {
        this.votesDB = new VotesDBImpl();
        this.vote = new Vote("Any");
        this.expected = new ArrayList<>();
    }

    @Test
    public void TestRegisterVote() throws Exception {
        this.votesDB.registerVote(this.vote);
        this.expected.add(this.vote);
        assertEquals(this.expected,this.votesDB.getVotes());
    }

    @Test
    public void TestGetVotes() throws Exception {
        for (int x=0; x < 3; x++) {
            this.votesDB.registerVote(this.vote);
            this.expected.add(this.vote);
        }
        assertEquals(this.expected, this.votesDB.getVotes());
    }

}
	\end{lstlisting}
	En aquesta classe comprovarem la correcta utilització i implementació de la base de dades de vots. Per aquest mètode crearem un vot i una llista de vots i registrarem un vot. Després comprovarem que la llista que ens retorni contingui el vot. Després farem el mateix per a més vots.	
\end{document}